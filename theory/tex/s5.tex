\lstset{language=C}
\section{Session 5}
\subsection{Constructing the mesh II}
\paragraph{}In the last session we worked with three different \texttt{.stl} files but we can also put all of these surfaces in a single file. To do so, we must modify the one of the files in order to append all he geometries. The Pdf \texttt{pipe.pdf} which can be found in atenea explains all the process.
\begin{enumerate}
	\item Put all the solids into a single file
	\item Modify SnappyHexMesh dictionary to specify these solids 	into the regions section.
\end{enumerate}
\paragraph{}The process is nearly the same as in the session 4. When executing a command we can print the terminal output of the command to a log (e.g \texttt{snappyHexMesh > snappy.log} or \texttt{foamJob -s snappyHexMesh, this one prints also the output to the terminal}. we can clean the polymesh folder directy by tipping \texttt{foamCleanPolyMesh}. If we want to only have a single mesh stored in the constant folder we can do it by tipping \texttt{snappyHexMesh -owerwrite}. It's important to have the final mesh stored in the constant folder.

\subsection{Simulating the case}
\paragraph{}A good advice before simulating a case is to make a copy of the 0 folder in order to be able to reset the case.

\paragraph{}Now we are going to edit the \texttt{0} inside the 0.orig folder to put our boundary conditions. Inside the boundary section of this file we have to modify the code to appear as following.
inlet1 0 -1 0
inlet2 -10 0 0

\begin{lstlisting}
internalField uniform(0 0 0);

.... %delete all the include lines under initialConditions

patchName
{
	type		fixedValue;
	value 		uniform(Ux Uy Uz);
}
\end{lstlisting}

\paragraph{}the \texttt{inletOutlet;} option in the type field means that in the oultet when the flow is going out of the domain the boundary condition is Zero Gradient avoiding the flow to returning inside of the domain.
\paragraph{}Now the computer know the velocity on different points of the geometry. And the computer will compute the relative pressure between the points on the domain. Hence, it has to know the pressure in one of the points. We modify the \texttt{p} file to have pressure equal to 0 in outlet and a zerogradient in the rest. Then we have to modify the file turbulenceProperties to perform a laminar simulation. 
